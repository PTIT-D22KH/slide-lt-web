\documentclass[../BTL.tex]{subfiles}
\begin{document}
\section{Kết luận}
Nhóm đã hoàn thành việc xây dựng một hệ thống website đơn giản, và đã triển khai được trên \href{https://epl-web-fe.onrender.com/}{Render(phần frontend)} và \href{https://epl-web.onrender.com/}{Render(phần backend)}. Kết quả của BTL nhóm em là một website với những chức năng cơ bản và một số chức năng thống kê đơn giản. Mã nguồn của phần backend BTL nhóm em ở trên \href{https://github.com/PTIT-Projects/football-stats-website-be}{Github}, và mã nguồn phần frontend BTL nhóm em ở trên \href{https://github.com/PTIT-Projects/football-stats-website-fe}{Github}. Ngoài ra, kết quả Docker image được đóng gói để triển khai trên Render cũng được lưu trên Dockerhub, với image của \href{https://hub.docker.com/repository/docker/vucongtuanduong/epl-web-fe/general}{frontend} và image của \href{https://hub.docker.com/repository/docker/vucongtuanduong/epl-web/general}{backend}
\section{ Hướng phát triển}
Như đã trình bày ở phần các chức năng chưa làm được, nhóm em có định hướng phát triển BTL web như sau: tiếp tục triển khai các tính năng chưa làm được như: 
\begin{itemize}
    \item Sửa đổi thông tin cá nhân tài khoản
    \item Tự động lấy dữ liệu từ những trang web cung cấp dữ liệu
    \item Những thống kê phức tạp hơn liên quan đến cầu thủ
    \item Báo cáo và xuất dữ liệu
    \item Tối ưu hóa hệ thống
\end{itemize}
Và cuối cùng là xây dựng được một CI/CD pipeline để có thể tự động hoá việc xây dựng, kiểm thử và triển khai trên cloud. Tuy nhiên đây là một công việc khó và đòi hỏi nhiều thời gian cũng như kiến thức để thực hiện

\end{document}